% \iffalse meta-comment
% Copyright © 2009 by Christophe Rhodes
%
% This file may be distributed and/or modified under the
% conditions of the LaTeX Project Public License, either version 1.3
% of this license or (at your option) any later version.
% The latest version of this license is in
%   http://www.latex-project.org/lppl.txt
% and version 1.3 or later is part of all distributions of LaTeX
% version 2005/12/01 or later.
% 
% This work has the LPPL maintenance status `maintained'; the current
% maintainer of this work, consisting of the files gsexam.ins and
% gsexam.dtx, is Christophe Rhodes <c.rhodes@gold.ac.uk>.
%
% \fi
%
% \iffalse
%<*driver>
\ProvidesFile{gsexam.dtx}
%</driver>
%<class|package>\NeedsTeXFormat{LaTeX2e}[1995/12/01]
%<class>\ProvidesClass{gsexam}
%<package>\ProvidesPackage{gsexam}
%<*class|package>
  [2009/07/22 v0.3 Goldsmiths Exam]
%</class|package>
%<*driver>
\documentclass{ltxdoc}
\EnableCrossrefs
\CodelineIndex
\RecordChanges
\begin{document}
\DocInput{gsexam.dtx}
\end{document}
%</driver>
% \fi
% 
% \changes{v0.1}{2009/03/27}{Initial version}
% \changes{v0.2}{2009/07/17}{Conversion to Documented \TeX}
% \changes{v0.3}{2009/07/22}{Implementation of rubric generation}
% \GetFileInfo{gsexam.dtx}
%
% \title{The \textsf{gsexam} package and class\thanks{This document
%     corresponds to \textsf{gsexam}~\fileversion, dated \filedate.}}
% \author{Christophe Rhodes}
%
% \maketitle
%
% \begin{abstract}
%   This document describes the \LaTeX-based software written to
%   support typesetting examinations for Goldsmiths, University of
%   London.  The software consists of a class file, intended to
%   typeset a single examination, and a style file which can be used
%   with other document classes, which can be used to typeset
%   collections of examinations, or examinations as a part of other
%   documents (such as External System subject guides).
% \end{abstract}
% 
% \section{Introduction}
%
% The task of typesetting examination papers is at least partly a
% mechanical one: the overall page layout is fairly rigidly
% determined; each question has a certain number of marks allocated to
% it; and so on.  Even the front-page rubrics have a standard form.
% Given all this, it would seem to make sense to capture the
% commonalities programmatically.  In addition, the codification of
% standard practice in \LaTeX\ specifically allows us to use the
% features provided by a Turing-complete programming language to
% automate menial tasks, such as ensuring that the marks allocated to
% question sub-parts sum to the total number of marks allocated to
% that question.
%
% Motivated by these general aims, and by the specific observation by
% the External Examiners of the Department of Computing for the
% 2008-09 session that there was potential for student confusion given
% the large number of variant formats in the examination papers
% submitted for scrutiny, the \textsf{gsexam} class and style file
% documented here is an attempt to allow examiners to set papers with
% the menial aspects of that setting taken care of, allowing them to
% concentrate on the task of setting an examination that is passable
% by all who have put in effort, yet can clearly discriminate between
% the good and the merely adequate.
%
% In addition, since examination papers are used in contexts other
% than for students to sit (for example, in External System subject
% guides), this software separates out the treatement of an
% examination as a collection of questions -- implemented in the
% \textsf{gsexam} style file, which can be used in a document of
% arbitrary \LaTeX\ class with |\usepackage| -- and an examination as
% a document to present to students, where certain niceties such as
% instructions to {\bf TURN~OVER} and page numbering to assure readers
% that no pages are missing are \textit{de rigeur}.
%
% The software described here has drawn from style files and \LaTeX\
% fragments written and maintained by Sebastian Danicic and Geraint
% Wiggins, to whom the author extends thanks for providing inspiration
% and a cross-section of existing practice.
%
% \section{User Guide}
%
% \subsection{Class or package?}
%
% The normal intended use for this software is to typeset an
% individual exam, whether that be for the normal session, the late
% Summer resit session, a sample paper (for a new or revised course),
% or for the External System of the University of London's Summer
% session.  In those circumstances, users should use the
% \textsf{gsexam} \emph{class}, by providing |gsexam| as the argument
% to the |\documentclass| \LaTeX\ directive.
%
% Under certain circumstances, such as the provision of a sample
% examination paper within a University of London External System
% Subject Guide, the \textsf{gsexam} \emph{package} or `style file'
% should be used instead, using |\usepackage{gsexam}| in the preamble,
% which gives access to the principal environments and macros, but
% does not override document headers and footers, or choose to include
% the contents of |answer| environments (as described in section
% \ref{s:answerinc} below)
%
% \subsection{The cover page}
%
% \DescribeEnv{bscexam} The |bscexam| environment encloses all of the
% information related to a single examination paper.  The environment
% takes a single, optional argument (currently defaulting to
% |onepart,nocalculators|), which should be a comma-separated list of
% keywords describing the exam.  The currently-accepted keywords are:
% \begin{itemize}
% \item |onepart|: the examination is a full-unit exam in one part;
% \item |twoparts|: the examination is a full-unit exam in two parts;
% \item |halfunit|: the examination is a half-unit exam;
% \item |calculators|: calculators (though not algebraic ones) are
%   allowed;
% \item |nocalculators|: no calculators are allowed;
% \item |practical|: the examination has a practical component.
% \end{itemize}
% 
% Within the |bscexam| environment, a number of macros are used to
% declare the metadata for a given paper:
% \DescribeMacro{\coursecode}|\coursecode| for the Goldsmiths and
% Departmental course codes;
% \DescribeMacro{\coursetitle}|\coursetitle| for the name of the
% course; \DescribeMacro{\examtime}|\examtime| for the normal length
% of the exam; \DescribeMacro{\examyear}|\examyear| for the year in
% which the exam is to be taken; and
% \DescribeMacro{\subject}|\subject| for the overall programme of
% study of the exam.  (These names for these metadata macros are not
% perfect, and may change in the near future.)
%
% The keywords given as the argument to the |bscexam| environment are
% used to generate a rubric in a standard form automatically.  If for
% whatever reason the automatically-generated rubric is not
% appropriate, the \DescribeMacro{\rubric}|\rubric| command can be
% used to specify a rubric for the examination.
%
% After all of these metadata fields are specified, the
% \DescribeMacro{\coverpage}|\coverpage| macro sets the cover page of
% the exam.
%
% \subsection{Questions and answers}
%
% \DescribeEnv{question} The |question| environment takes two
% arguments.  The first, the title of the question, is optional (and
% so if present is given in square brackets, as is usual in \LaTeX).
% The second argument is mandatory, and is the number of marks
% allocated for that question.
%
% The |question| environment implicitly begins an enumerated list,
% such that \DescribeMacro{\item}|\item| within it begins a question
% part.  It is also possible to have a question with no parts at all.
%
% It is possible to use the normal \LaTeX\ handling of references to
% refer to particular question parts: each part to be referred to
% should have a unique \DescribeMacro{\label}|\label| attached to it,
% and correcly-formatted references can then be generated (with the
% usual caveat that two passes of the \LaTeX\ processor might be
% needed) using \DescribeMacro{\ref}|\ref|.
%
% Within a question, the \DescribeMacro{\qmarks}|\qmarks| macro will
% set, within square brackets and in the right margin, the number of
% marks given as its argument.  In addition, unless the |unchecked|
% option has been given, the total number of marks given to |\qmarks|
% within a question is compared against the declared number of marks
% given to that |question| environment, and if there is a mismatch, a
% diagnostic message is emitted.
%
% \DescribeEnv{answer} The |answer| environment is for providing the
% answers to questions.  Under default operation, the contents of the
% |answer| environment are simply discarded; however, using the
% |answers| class option, or by other means (see section
% \ref{s:answerinc} below) the contents of the |answer| environment
% will be typeset, in place, in the italic shape font for the
% document.
%
% \subsubsection{Default inclusion of \texttt{answer} environments}
% \label{s:answerinc}
%
% If using the \textsf{gsexam} style only, the default treatment of
% |answer| environments is to exclude them, including them only with
% an explicit |answers| argument in the |\usepackage{gsexam}| line:
% effectively the default is the same treatement as passing an
% explicit |noanswers| argument.
%
% However, if using the \textsf{gsexam} class file, as when
% typesetting an individual exam, the default treatment is slightly
% different: in all cases an explicit |answers| or |noanswers|
% argument is respected; however, when both are absent the default is
% not to include answers, \emph{except} if the \TeX\ |\jobname|
% variable contains the string `|answers|'.  The |\jobname| itself
% defaults to the name of the |.tex| file being processed, but this
% can be overridden using the |-jobname| command-line flag.  This
% behaviour allows the production of versions of the exam with and
% without answers without editing the \LaTeX\ source file.
%
% \subsubsection{Editor support for the \texttt{answer} environment}
%
% The |answer| environment is implemented using the |comment| package
% (see section \ref{s:answerimp} for details), which itself is based
% on the |verbatim| package.  This means that the |answer| environment
% only works absolutely properly if |\begin{answer}| and
%   |\end{answer}| are the first things on their respective lines.
%
% The |gsexam.el| file which accompanies this software is sufficient
% to inform AUC\-\TeX (the emacs mode for editing \LaTeX) about the
% need to indent |\begin{answer}| and |\end{answer}| specially.  It
% can be loaded directly, or will be dealt with automatically if
% installed in a directory mentioned in the emacs variable
% |TeX-style-path|.  (If editing a file from scratch, typing |C-c C-n|
% or selecting the |Reset Buffer| menu option from the |LaTeX| menu
% after the buffer contents contain the directive to load this class
% should be sufficient to load the relevant customizations).
%
% \StopEventually{\PrintChanges\PrintIndex}
%
% \section{Implementation}
%
% \subsection{The \textsf{gsexam} Package}
% \label{s:package}
%
% The \textsf{gsexam} package is intended to be used, with
% |\usepackage|, in any document where there is a need to include a
% Goldsmiths exam.  An example might be a booklet collecting all of a
% given year's papers, to give (or sell) to local students to help
% with their revision.  Another example is a Subject Guide produced
% for the University of London External Programme, which typically
% includes a sample examination paper.  There is an accompanying
% \textsf{gsexam} class file, intended for typesetting a single
% examination paper such that the output can be presented to students
% in an examination context, whose implementation is described in
% section \ref{s:class} below.
%
% \subsubsection{\textsf{gsexam} Package Options}
% \label{s:poptions}
%
%\iffalse
%<*package>
%\fi
%
% The \textsf{gsexam} package accepts a number of options, which can
% be provided through the use of |\usepackage|.  In addition, the same
% options are passed through from the class options if the
% \textsf{gsexam} document class is used.
% \begin{macro}{\if@answers}
% \begin{macro}{\@answerstrue}
% \begin{macro}{\@answersfalse}
%   The |answers| and |noanswers| package options to the
%   \textsf{gsexam} package set the |@answers| boolean to true and
%   false respectively, which will controls the effect of the
%   |answers| environment described in section \ref{s:answerimp} below.
%    \begin{macrocode}
\newif\if@answers
\DeclareOption{answers}{\@answerstrue}
\DeclareOption{noanswers}{\@answersfalse}
%    \end{macrocode}
% \end{macro}
% \end{macro}
% \end{macro}
%
% \begin{macro}{\if@checked}
% \begin{macro}{\@checkedtrue}
% \begin{macro}{\@checkedfalse}
%   The default behaviour of the |question| environment provided by
%   the \textsf{gsexam} package is to check the total number of marks
%   allocated for a question against the declared allocation.  The
%   |unchecked| package option turns this checking off globally, by
%   setting the |@checked| boolean to false.
%    \begin{macrocode}
\newif\if@checked
\@checkedtrue
\DeclareOption{unchecked}{\@checkedfalse}
%    \end{macrocode}
% \end{macro}
% \end{macro}
% \end{macro}
% 
% After declaring the options, we process them.
%    \begin{macrocode}
\ProcessOptions
%    \end{macrocode}
% 
% \subsubsection{The \texttt{answer} environment}
% \label{s:answerimp}
%
% The |answer| environment is, effectively, a conditional comment.  If
% the user has requested that answers should be included, then
% material within an |answer| environment should be typeset in some
% way; if a user has requested that answers not be included, then the
% material within an |answer| environment should be elided.  (The
% default setting for the |@answer| boolean which controls this is
% marginally complicated; see section \ref{s:answerinc} and the
% \textsf{gsexam} class documentation below.
%
% The |comment| package provides a simple mechanism to allow for
% custom comment environments to be defined.
%    \begin{macrocode}
\RequirePackage{comment}
%    \end{macrocode}
% If the |@answers| boolean is true, we include the |answer| comments
% in italics, beginning with a paragraph.  If the |@answers| boolean
% is false, we simply exclude the contents of the |answer|
% environment.
%    \begin{macrocode}
\if@answers
  \specialcomment{answer}{\begingroup\itshape\par}{\endgroup}
\else
  \excludecomment{answer}
\fi
%    \end{macrocode}
%
% \subsubsection{The \texttt{question} environment}
%
% We define a |question| counter for enumerating questions.
%    \begin{macrocode}
\newcounter{question}
%    \end{macrocode}
%
% \begin{macro}{\qmarks}
%   We define a |qmarks| counter to tally the marks allocated for each
%   question.  The |\qmarks| macro is responsible not only for
%   typesetting the number of marks allocated for a part of a
%   question, but also for incrementing the |qmarks| counter by the
%   appropriate amount.
%    \begin{macrocode}
\newcounter{qmarks}[question]
\newcommand{\qmarks}[1]{%
  \unskip\nobreak\hfill\rlap{\phantom{00000}\llap{[#1]}}%
  \addtocounter{qmarks}{#1}}
%    \end{macrocode}
% \end{macro}
% The |question| environment typesets an individual question.  It
% takes two arguments: an optional question title (defaulting to
% empty), and a mandatory number of marks.
%    \begin{macrocode}
\newenvironment{question}[2][]{
%    \end{macrocode}
% Within a |question| environment, we ensure that labels and
% references to enumerated |\item|s are appropriate: the labels are of
% the form ``(a)'' and ``ii.'', while references generated with |\ref|
% are ``(a)'' and a.(ii) repectively for first- and second-level
% |\item|s.
%    \begin{macrocode}
  \def\theenumi{(\alph{enumi})}
  \def\labelenumi{\theenumi}
  \def\theenumii{\alph{enumi}.(\roman{enumii})}
  \def\labelenumii{\roman{enumii}.}
%    \end{macrocode}
% We store the stated number of marks for this question in a local
% variable.
%    \begin{macrocode}
  \def\qm@rks{#2}
%    \end{macrocode}
% We step the |question| counter, which also resets the |qmarks|
% counter to zero
%    \begin{macrocode}
  \stepcounter{question}
%    \end{macrocode}
% The question itself is set as a |trivlist| with one item: the title,
% if any, is set on the same line as the \textbf{Question} heading.
%    \begin{macrocode}
  \begin{trivlist}
    \item[\bfseries Question \arabic{question}] {\qquad#1}
      \begin{enumerate}
%    \end{macrocode}
% Within this and child |enumerate| environments, we request a separation 
% of the text from the label by 0.5em.  Without this, the label and text 
% abut directly on each other.
%    \begin{macrocode}
        \setlength{\labelsep}{.5em}
  }
  {\end{enumerate}\end{trivlist}%
%    \end{macrocode}
% At the end of the question, we check whether the value of the
% |qmarks| counter is the same as the declared number of marks for
% this question (stored in |\qm@arks|)
% \begin{macrocode}
\ifnum\qm@rks=\value{qmarks}
%    \end{macrocode}
% If it is, we do nothing.  Otherwise, we print a diagnostic message
% to the output stream, and typeset a warning about the mark count
% mismatch at the end of the question.
%    \begin{macrocode}
\else
  \if@checked
  \typeout{Mark mismatch in question \arabic{question}: 
    expected \qm@rks, got \arabic{qmarks}}
  \centering 
  This question has a mark count mismatch: expected \textbf{\qm@rks},
  got \textbf{\arabic{qmarks}}.\par
  \fi
\fi}
%    \end{macrocode}
%
% \subsubsection{The \texttt{bscexam} environment}
%
% \begin{macro}{\if@examend}
% \begin{macro}{\@examendtrue}
% \begin{macro}{\@examendfalse}
% We define a boolean to keep track of whether we have got to the end
% of the exam or not.  It is reset (to false) at the beginning of each
% |bscexam| environment.  At present, it's only used for generating
% the appropriate page footer (which is handled by the
% \textsf{gsexam} class).
%    \begin{macrocode}
\newif\if@examend
%    \end{macrocode}
% \end{macro}
% \end{macro}
% \end{macro}
%
% \begin{macro}{\coursecode}
% \begin{macro}{\coursetitle}
% \begin{macro}{\examtime}
% \begin{macro}{\examyear}
% \begin{macro}{\subject}
%   We also provide a number of commands to store metadata for a
%   particular exam, which simply store their respective arguments in
%   an internal macro for later use.  Note that the names of these
%   commands are perhaps not ideal, given the Goldsmiths internal and
%   the UoL External System nomenclatures: it is possible that
%   |\subject| will be renamed to |\programme|, or similar.
%    \begin{macrocode}
\newcommand{\coursecode}[1]{\def\@coursecode{#1}}
\newcommand{\coursetitle}[1]{\def\@coursetitle{#1}}
\newcommand{\examyear}[1]{\def\@examyear{#1}}
\newcommand{\subject}[1]{\def\@subject{#1}}
\newcommand{\examtime}[1]{\def\@examtime{#1}}
%    \end{macrocode}
% \end{macro}
% \end{macro}
% \end{macro}
% \end{macro}
% \end{macro}
%
% \begin{macro}{\rubric}
%   \changes{v0.3}{2009/07/22}{Added.}
%
%   The |\rubric| macro is used to supress the automatic generation of
%   a rubric from keywords, instead causing its argument to be the
%   entirety of the rubric (except for the prohibition of removal of
%   the exam paper from the room).
%    \begin{macrocode}
\newcommand{\rubric}[1]{
  \def\rubric@kind{}
  \def\rubric@calc{}
  \def\rubric@explicit{#1}}
%    \end{macrocode}
% \end{macro}
%
% \begin{macro}{\coverpage}
%   \changes{v0.3}{2009/07/22}{Implemented rubrics.}
%
%   The |\coverpage| macro produces the cover page.  The |\@coverhead|
%   macro takes care of the header information, while the rest of the
%   code typesets information specific to the kind of paper it is and
%   whether calculators are allowed, or alternatively an explicit
%   rubric given before |\coverpage| in the |bscexam| environment.
%    \begin{macrocode}
\newcommand{\coverpage}{\@coverhead%
  \begin{list}{}{
      \setlength{\parsep}{2ex}
      \setlength{\parindent}{0pt}
      \setlength{\leftmargin}{0pt}
      \setlength{\topsep}{5ex}}
  \item
    {\itshape\rubric@kind\par}
    {\itshape\rubric@calc\par}
    {\itshape\rubric@explicit\par}
    \begin{center}
      \textbf{THIS PAPER MUST NOT BE REMOVED\\
        FROM THE EXAMINATION ROOM}
    \end{center}
  \end{list}\clearpage}
%    \end{macrocode}
% \end{macro}
%
% \begin{macro}{\rubric@onepart}
% \begin{macro}{\rubric@twoparts}
% \begin{macro}{\rubric@halfunit}
%   The |\rubric@onepart|, |\rubric@twoparts| and |\rubric@halfunit|
%   macros contain the standard rubric texts for a full-unit
%   single-part exam, a full-unit two-part exam, and a half-unit exam
%   respectively.
%    \begin{macrocode}
\newcommand{\rubric@onepart}{There are six questions in this paper.
  You should answer no more than four questions.  Full marks will be
  awarded for complete answers to a total of four questions.  Each
  question carries 25 marks.  The marks for each part of a question
  are indicated at the end of the part in \textup{[.]} brackets.\par
  There are 100 marks available on this paper.}
\newcommand{\rubric@twoparts}{This paper is in two parts: part A and
  part B.  There are a total of three questions in each part.  You
  should answer two questions from part A and two questions from part
  B.  Your answers to part A and part B should be written in separate
  answer books.\par Full marks will be awarded for complete answers to
  a total of four questions, two from part A and two from part B.
  Each question carries 25 marks.  The marks for each part of a
  question are indicated at the end of the part in \textup{[.]}
  brackets.\par
  There are 100 marks available on this paper.}
\newcommand{\rubric@halfunit}{There are five questions in this paper.
  You should answer no more than three questions.  Full marks will be
  awarded for complete answers to a total of three questions.  Each
  question carries 25 marks.  The marks for each part of a question
  are indicated at the end of the part in \textup{[.]} brackets.\par
  There are 75 marks available on this paper.}
%    \end{macrocode}
% \end{macro}
% \end{macro}
% \end{macro}
%
% \begin{environment}{bscexam}
%   \changes{v0.3}{2009/07/22}{Added optional argument}
%
%   The |bscexam| environment surrounds a number of |question|
%   environments.  It takes a single optional argument (defaulting to
%   |onepart,nocalculators|), which should be a comma-separated list
%   of keywords describing the kind of exam; these keywords are used
%   to generate the rubric.  Exams not in the standard format can
%   disable the automatic generation of a rubric, and provide one
%   manually using the |\rubric| macro.
%    \begin{macrocode}
\newenvironment{bscexam}[1][onepart,nocalculators]{
%    \end{macrocode}
% At the start of the exam, we ensure that the |\@examend| boolean
% is set to false.
%    \begin{macrocode}
  \@examendfalse
%    \end{macrocode}
% Exams start on a new page.
%    \begin{macrocode}
  \clearpage
%    \end{macrocode}
% The first question is question 1.
%    \begin{macrocode}
  \setcounter{question}{0}
%    \end{macrocode}
% We set some list-related lengths (the margins for the first two
% enumeration levels, and the label width) to sensible values here.
% These length definitions are local to the |bscexam| environment
%    \begin{macrocode}
  \setlength{\leftmargini}{2.5em}
  \setlength{\leftmarginii}{1.5em}
  \setlength{\labelwidth}{\leftmargini}
%    \end{macrocode}
%    \begin{macrocode}
  \def\rubric@kind{}
  \def\rubric@calc{}
  \def\rubric@explicit{}
  \@for\opt@temp:=#1\do{
    \def\opt@foo{onepart}
    \ifx\opt@temp\opt@foo
    \def\rubric@kind{\rubric@onepart}
    \fi
    \def\opt@foo{twoparts}
    \ifx\opt@temp\opt@foo
    \def\rubric@kind{\rubric@twoparts}
    \fi
    \def\opt@foo{halfunit}
    \ifx\opt@temp\opt@foo
    \def\rubric@kind{\rubric@halfunit}
    \fi
    \def\opt@foo{calculators}
    \ifx\opt@temp\opt@foo
    \def\rubric@calc{Electronic calculators must
      not be programmed prior to the examination.  Calculators which
      display graphics, text or algebraic equations are not allowed.}
    \fi 
    \def\opt@foo{nocalculators}
    \ifx\opt@temp\opt@foo
    \def\rubric@calc{No calculators should be used.}
    \fi
    \def\opt@foo{practical}
    \ifx\opt@temp\opt@foo
    \def\rubric@calc{This is a practical examination; each answer
      requiring code or other computational material should be named
      according to question number, part and sub-part: for example,
      \textup{\texttt{Q5\_b\_2.pde}} for a Processing sketch in answer
      to part (b) sub-part (ii) of question 5.  Save your answer to
      the exam submission folder.  You are responsible for ensuring
      that your answers have been saved in the correct location.}
    \fi
  }
%    \end{macrocode}
% 
% \begin{macro}{\@coverhead}
%   The |\@coverhead| macro, local to the |bscexam| environment,
%   produces the top half of the cover page of an exam, with all the
%   relevant fields formatted.
%    \begin{macrocode}
  \def\@coverhead{
    \begin{flushleft}
      \Large \bfseries UNIVERSITY OF LONDON\\ [14pt]
      GOLDSMITHS COLLEGE\\ [14pt]
      B.~Sc.~Examination \@examyear\\ [28pt]
      \@subject\\ [14pt]
      \@coursecode \quad \@coursetitle\\ [14pt]
      \large \bfseries Duration: \@examtime\\ [14pt]
      Date and time: \\ [14pt]
    \end{flushleft}
    \hrule\par\vskip12pt\relax}}
%    \end{macrocode}
% \end{macro}
%
% At the end of the exam, we set |\@examend| to be true.
%    \begin{macrocode}
  {%
    \@examendtrue%
%    \end{macrocode}
% Then we clear the page, so that the any footer (such as that coming 
% from the \textsf{gsexam} class with {\bf END OF EXAMINATION}) is set
% within the scope of the |bscexam| environment.
%    \begin{macrocode}
    \clearpage}
%    \end{macrocode}
% \end{environment}
% 
%\iffalse
%</package>
%\fi
%
% \subsection{The \textsf{gsexam} Class}
% \label{s:class}
%
%\iffalse
%<*class>
%\fi
%
% The use of |\IfSubStringInString| below requires e\TeX, which we
% therefore test for here.
%    \begin{macrocode}
\ifx\eTeXversion\@undefined
  \ClassError{gsexam}
  {This class requires eTeX}
  \expandafter\@@end
\fi
%    \end{macrocode}
% \subsubsection{Class options}
%
% We need the |substr| package to implement our own class option handling
%    \begin{macrocode}
\RequirePackage{substr}
%    \end{macrocode}
% 
% The |unchecked|, |answers| and |noanswers| class options are simply
% passed straight through to the \textsf{gsexam} package, as described in
% section \ref{s:poptions}.
%    \begin{macrocode}
\DeclareOption{unchecked}{\PassOptionsToPackage{unchecked}{gsexam}}
\DeclareOption{answers}{\PassOptionsToPackage{answers}{gsexam}}
\DeclareOption{noanswers}{\PassOptionsToPackage{noanswers}{gsexam}}
%    \end{macrocode}
% We provide |oneside| and |twoside| class options, defaulting to
% two-sided layout.
%    \begin{macrocode}
\@twosidetrue
\DeclareOption{oneside}{\@twosidefalse}
\DeclareOption{twoside}{\@twosidetrue}
%    \end{macrocode}
% 
% We also implement the default handling of the |answer| environment
% by dispatching on the |\jobname| using the |substr| package.
%    \begin{macrocode}
\IfSubStringInString{\detokenize{answers}}{\jobname}%
  {\PassOptionsToPackage{answers}{gsexam}}%
  {}
%    \end{macrocode}
% That's all the options, which can now be processed.
%    \begin{macrocode}
\ProcessOptions
%    \end{macrocode}
%
% \subsubsection{Implementation of functionality}
%
% The minimal requirements for a class are to provide a text height
% and width, a definition for |\normalsize|, and a page numbering
% directive, which we do here.  We must do this after the options have
% been processed, in order that the |\@twoside| boolean has been set
% as the user requests.
%    \begin{macrocode}
\setlength\paperheight{297mm}
\setlength\paperwidth{210mm}
\input{size11.clo}
\pagenumbering{arabic}
%    \end{macrocode}
%
% We require a couple of other packages: |lastpage| to help with
% setting page numbers in the form ``page $m$ of $n$'', and |geometry|
% to set tolerably narrow margins (overriding those that were
% generated from |size10.clo|)
%    \begin{macrocode}
\RequirePackage{lastpage}
\RequirePackage[lmargin=1.25in,rmargin=1.25in]{geometry}
%    \end{macrocode}
% Finally, we also require our own \textsf{gsexam} package, which
% implements the various environments.
%    \begin{macrocode}
\RequirePackage{gsexam}
%    \end{macrocode}
%
% Finally comes the implementation of the default header and footer
% layout for exams.  By default, there's nothing in the headers.
%    \begin{macrocode}
\def\@evenhead{}
\def\@oddhead{}
%    \end{macrocode}
% The footers contain examination information.  The left of the footer
% contains a reference to the course code and the year of the
% examination; the centre contains page numbering references (in the
% form "page m of n", in an attempt to make it very clear whether
% there are any pages missing); and the right contains \textbf{END OF
%   EXAMINATION} at the end of an exam, and otherwise \textbf{TURN
%   OVER} on all right-hand pages (which is \emph{all} pages if the
% |oneside| class option is provided).
%    \begin{macrocode}
\def\@leftmark{\makebox[0in][l]{\bfseries \@coursecode\quad\@examyear}}
\def\o@rightmark{\makebox[0in][r]{\bfseries%
    \if@examend END OF EXAMINATION\else TURN OVER\fi}}
\if@twoside
  \def\e@rightmark{\makebox[0in][r]{\bfseries%
      \if@examend END OF EXAMINATION\fi}}
\else
  \def\e@rightmark\o@rightmark
\fi
\def\@evenfoot%
  {\@leftmark \hfil%
    page \thepage\ of \pageref{LastPage} \hfil%
    \e@rightmark}
\def\@oddfoot%
  {\@leftmark \hfil%
    page \thepage\ of \pageref{LastPage} \hfil%
    \o@rightmark}
%    \end{macrocode}
%\iffalse
%</class>
%\fi
%
% \Finale