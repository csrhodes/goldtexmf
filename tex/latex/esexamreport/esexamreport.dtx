% \iffalse meta-comment
% Copyright © 2009 by Christophe Rhodes
%
% This file may be distributed and/or modified under the
% conditions of the LaTeX Project Public License, either version 1.3
% of this license or (at your option) any later version.
% The latest version of this license is in
%   http://www.latex-project.org/lppl.txt
% and version 1.3 or later is part of all distributions of LaTeX
% version 2005/12/01 or later.
% 
% This work has the LPPL maintenance status `maintained'; the current
% maintainer of this work, consisting of the files esexamreport.ins
% and esexamreport.dtx, is Christophe Rhodes <c.rhodes@gold.ac.uk>.
% \fi
%
% \iffalse
%<*driver>
\ProvidesFile{esexamreport.dtx}
%</driver>
%<class>\NeedsTeXFormat{LaTeX2e}[1999/12/01]
%<class>\ProvidesClass{esexamreport}
%<*class>
  [2009/07/05 v0.0 UoL External System Examiners' Report]
%</class>
%<*driver>
\documentclass{ltxdoc}
\EnableCrossrefs
\CodelineIndex
\RecordChanges
\begin{document}
\DocInput{esexamreport.dtx}
\end{document}
%</driver>
% \fi
%
% \changes{v0.0}{2009/07/05}{Initial version}
%
% \GetFileInfo{esexamreport.dtx}
%
% \title{The \textsf{esexamreport} class\thanks{This document
%     corresponds to \textsf{esexamreport}~\fileversion, dated
%     \filedate.}}
% \author{Christophe Rhodes}
%
% \maketitle
%
% \begin{abstract}
%   This class file codifies the typographic requirements for
%   Examiners' Reports for the University of London External System.
% \end{abstract}
%
% \section{Introduction}
%
% The typographical requirements for Examiners' Reports for the
% University of London External System are surprisingly strict, and
% (let it be said) oriented towards a simple and straightforward use
% of word processors such as Microsoft Word.  This \LaTeX\ class file
% was written for the use of those who are unable or unwilling to use
% such software.
%
% \section{User Guide}
%
% The use of this class is simple; there are currently no class
% options.
%
% \DescribeMacro{\coursecode} \DescribeMacro{\coursetitle} The user
% defines the course code and course title for which the report is
% written, using the |\coursecode| and |\coursetitle| macros in the
% preamble; the exam zone is likewise specified using the |\zone|
% \DescribeMacro{\zone} macro in the preamble.
%
% As the first element in the main matter, \DescribeMacro{\maketitle}
% |\maketitle| will create a suitable title from the metadata given.
%
% Following this, the examiners' report should be written, using the
% \DescribeMacro{\heading} |\heading| macro to produce section
% headings according to the style guide given.
%
% \StopEventually{\PrintChanges\PrintIndex}
%
% \section{Implementation}
%
% We piggy-back on the \textsf{article} class.  In future, we may
% instead use something more like the \textsf{minimal} class: any use
% of any features provided by the \textsf{article} class is
% unsupported.
%    \begin{macrocode}
\LoadClass[a4paper,12pt]{article}
%    \end{macrocode}
% The style guide of the University of London imposes Times on us.
%
%  FIXME: we should make this XeTeX-friendly too, using Times New
%  Roman instead of Type 3 Times fonts.
%    \begin{macrocode}
\usepackage{times}
%    \end{macrocode}
% We use the \textsf{geometry} package to adjust the margins,
% accepting the defaults of that package.
%    \begin{macrocode}
\usepackage{geometry}
%    \end{macrocode}
% \begin{macro}{\coursecode}
% \begin{macro}{\coursetitle}
% \begin{macro}{\zone}
% The commands for setting metadata simply store their arguments in
% internal macro definitions:
%    \begin{macrocode}
\newcommand{\coursecode}[1]{\def\@coursecode{#1}}
\newcommand{\coursetitle}[1]{\def\@coursetitle{#1}}
\newcommand{\zone}[1]{\def\@zone{#1}}
%    \end{macrocode}
% \end{macro}
% \end{macro}
% \end{macro}
% \begin{macro}{\maketitle}
% and those internal macros are used in our implementation of
% |\maketitle|
%    \begin{macrocode}
\renewcommand{\maketitle}{%
  \begin{flushleft}
    Examiners' Report \number\year\\
    \@coursecode\ \@coursetitle\\
    Zone \@zone
  \end{flushleft}}
%    \end{macrocode}
% \end{macro}
% \begin{macro}{\heading}
% The formatting specified for headings and body text is to make the
% headings bold, and to separate paragraphs.  At the moment, we are
% using a sectioning command from \textsf{article} (specifically
% |\subsubsection*|), but this will probably change in future, as it
% causes the implementation to be slightly twisted (because the
% |\subsubsection*| itself generates vertical space, which needs to be
% cancelled).  This gives rise to an odd visible artifact if two
% |\heading|s follow each other with no intervening text.
%    \begin{macrocode}
\newcommand{\heading}[1]{%
  \setlength{\parskip}{0ex}%
  \subsubsection*{#1}%
  \setlength{\parskip}{2ex}%
  \vspace{-2ex}}
%    \end{macrocode}
% \end{macro}
% Finally, the style guide also specifies zero paragraph indent.
%    \begin{macrocode}
\setlength{\parindent}{0pt}
%    \end{macrocode}
%
% \Finale